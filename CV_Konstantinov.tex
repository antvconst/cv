\documentclass[11pt,a4paper,sans]{moderncv}
\usepackage[utf8]{inputenc}
\usepackage[english, russian]{babel}
\moderncvstyle{classic}
\moderncvcolor{blue}
\usepackage[scale=0.87]{geometry}
\setlength{\hintscolumnwidth}{2.5cm}

\firstname{Антон}
\familyname{Константинов}

\phone{+7(952)359-22-55}
\email{antv.konstantinov@gmail.com}
\extrainfo{GitHub: \href{http://github.com/falceeffect}{falceeffect}}
\nopagenumbers{}

\begin{document}
\makecvtitle

\section{Образование}
\cventry{2015--2019}{Бакалавр по специальности <<01.03.02 Прикладная математика и информатика>>}{\newline СПбГУ}{\newline \textit{Кафедра}: Cтатистического моделирования}{}{}
\cventry{2017--Настоящее время}{Студент}{Computer Science Center}{\newline \textit{Направление}: Data Science}{}{}

\section{MOOC}
\cventry{2017}{Algorithmic Toolbox by University of California, San Diego \& National Research University Higher School of Economics on Coursera}{\href{https://www.coursera.org/account/accomplishments/certificate/MUXTS2U889ZD}{подтверждённый сертификат}}{}{}{}
\cventry{2017}{Data Structures by University of California, San Diego \& National Research University Higher School of Economics on Coursera}{\href{https://www.coursera.org/account/accomplishments/certificate/HF284YQVC9RM}{подтверждённый сертификат}}{}{}{}
\cventry{2017}{Algorithms on Graphs by University of California, San Diego \& National Research University Higher School of Economics on Coursera}{\href{https://www.coursera.org/account/accomplishments/certificate/FF4KB3G8LNXX}{подтверждённый сертификат}}{}{}{}
\cventry{2018}{Algorithms on Strings by University of California, San Diego \& National Research University Higher School of Economics on Coursera}{\href{https://www.coursera.org/account/accomplishments/certificate/6WQH8FPJXSNF}{подтверждённый сертификат}}{}{}{}
\cventry{2018}{Обучение на размеченных данных by Moscow Institute of Physics and Technology, Yandex \& ФРОО on Coursera}{\href{https://www.coursera.org/account/accomplishments/certificate/JWLKDHQVKEKW}{подтверждённый сертификат}}{}{}{}

\section{Внеучебная деятельность}
\cvline{2018}{Зимняя физико-математическая школа <<Абсолютное будущее>>, Москва, МФТИ}
\cvline{2017}{Студенческая олимпиада <<Я --- профессионал>>, секция: математика}

\section{Навыки}
\cvitem{OS}{GNU/Linux, Windows}
\cvitem{Языки прогр. и технологии}{Python, NumPy, SciPy, C++, Qt, R, Git}
\cvitem{Разное}{LaTeX}


\section{Интересы}
\cvitem{Математика}{Теория вероятностей, математическая статистика, анализ временных рядов}
\cvitem{Анализ данных}{Машинное обучение, нейронные сети}
\cvitem{Прогр.}{Разработка на Python, C++}
	
\section{Увлечения}
\cvitem{Музыка}{Гитара, фортепиано}
\section{Персональная информация}
\cvitem{Имя}{Константинов Антон Владимирович}{}{}{}{}
\cvitem{Место жительства}{Петергоф}{}{}{}{}
\cvitem{Дата рождения}{16.02.1997}{}{}{}{}
\cvitem{Языки}{Русский (родной), English (fluent)}{}{}{}{}


\end{document}